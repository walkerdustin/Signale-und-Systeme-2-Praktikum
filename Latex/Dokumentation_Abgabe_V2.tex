\newpage
\section{Aufgabe 2}
Zur Ermittlung der Position des Moskitos wird dessen Fluggeräusch mit vier Mikrofonen gleichzeitig abgetastet.
Dabei wird das Signal mit 48kHz bzw. 96kHz abgetastet. 
Der Empfänger zeichnet gleichzeitig die empfangenen Signale an allen vier Mikrofonen auf. 
Anhand der empfangenen Signalen werden die relativen Laufzeitverzögerungen zwischen den Mikrofonen berechnet und darüber anschließend die Position mit dem bereits erklärten "Newton-Verfahren" ermittelt.
Die relativen Verzögerungen werden dabei die empfangenen Signale mit einem aus dem Signal ausgeschnittenen Teilsignal korreliert. Durch die Korrelation eines Teilsignals mit den vier Hauptsignalen lässt sich jeweils die Position des Teilsignals in den Hauptsignalen ermitteln. Anhand der Position lässt sich wiederrum die relative Verzögerung ermitteln. Zur Vereinfachung werden hier vorerst lediglich die Signale von zwei Mikrofonen betrachtet.
\subsection{Herrauslösen des Teilsignals}
Da die Mikrofone zeitgleich das Signal des Moskitos aufzeichnen, kommt es aufgrund der Laufzeitunterschiede des Signals zu einer Verschiebung der Signalwerte in den empfangenen Signalen untereinander. \\
So empfängt beispielsweise MIC1 den Signalabschnitt $X$ nach 10ms, wohingegen MIC2 aufgrund der größeren Entfernung zum Moskito den selben Abschnitt $X$ erst nach 20ms empfängt.\\
Beginnt man nun beim Herrauslösen des Teilsignals am Anfang des Empfangsignals führt dies dazu, dass Mikrofone die näher am Moskito sind diese gar nicht empfangen haben. Dies hat zur Folge, dass die Laufzeitdifferenz nicht korrekt ermittelt werden kann, da die Korrelation keine korrekten Ergebnisse liefern kann.\\
Um diesem Fehlerfall entgegenzu wirken, darf das Teilsignal erst nach einem "Totbereich" herrausgelöst werden. Die Dauer des Totbereiches $t_{min}$ wird dabei über den maximal möglichen Abstand $a_{max}$ eines Mikrofons zum Moskito ermittelt:
$$	t_{min} = \frac{a_{max}}{c_{s}} $$
Über die $t_{min}$ und die SamplingRate $f_s$ lässt sich dabei die Größe des Totbereiches $K_{min}$ in der Indexierung des Singales errechnen:
\begin{equation}
	K_{min} = f_s * t_{min}   = f_s * \frac{a_{max}}{c_{s}} \label{eq:A2A2E1}
\end{equation}

Unter der Annahme dass sich das Moskito in einem $1m x 1m x 1m$-Raum aufhält, ergibt sich für den maximalen Abstand $a_{max}$ die Raumdiagonale mit einer Länge von $a_{max} = \sqrt{3}m$. Mit einer SamplingRate von $f_s = 96 kHz$ folgt aus der Gleichung\eqref{eq:A2A2E1} ein Totbereich von $K_{min} \approx 485$. Um noch etwas Sicherheit einzubauen wurde für das Matlab Programm ein Totbereich von $K_{min} = 500$ gewählt.

\subsection{Länge des Teilsignals}
Die Länge des Teilsignals beieinflusst die Genauigkeit der Differenzmessung deutlich. Ist die Länge des Teilsignals zu kurz, ist keine genaue Positionsermittlung möglich, da die Wahrscheinlichkeit, dass ein kurzes Teilsignal mehrmals in der Sequenz vorkommt deutlich höher ist. Durch die Verlängerung des Teilsignals wird die Wahrscheinlichkeit, ähnliche Sequenzen in einem Signal zu haben, deutlich reduziert. Längere Zeitsignale erhöhen den Rechenaufwand und damit die Rechendauer, dies hat zur Folge, dass nach dem Abschluss der Positionsbestimmung das Mosquito bereits an einer anderen Stelle im Raum befindet. \\
Daher ist es nötig ein geeingetes Zwischenmaß zu finden. Daher gilt es die Länge des Teilsignals somit die Anzahl der Korrelationswerte zu untersuchen.